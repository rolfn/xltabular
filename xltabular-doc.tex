%% $Id: xltabular-doc.tex 598 2017-09-30 19:52:25Z herbert $
%
\listfiles\setcounter{errorcontextlines}{100}
\documentclass[paper=a4,fontsize=11pt,DIV=13,parskip=half-,
               captions=tableabove,twoside=on]{scrartcl}
\usepackage{fontspec}
%\usepackage[%usefilenames,
%            TT={Scale=0.88,FakeStretch=0.9},
%            SS={Scale=0.9},
%            RM={Scale=0.9},
%            DefaultFeatures={Ligatures=TeX}]{lucida-otf}  % support opentype math fonts
\setmainfont{AccanthisADFStdNo3}[
  UprightFont   =*-Regular,
  BoldFont      =*-Bold,
  ItalicFont    =*-Italic,
  BoldItalicFont=*-BoldItalic,
  RawFeature    = -rlig,
]
\setsansfont{GilliusADF}[
  UprightFont   =*-Regular,
  BoldFont      =*-Bold,
  ItalicFont    =*-Italic,
  BoldItalicFont=*-BoldItalic,
  RawFeature    = -rlig,
]
\setmonofont{Anonymous Pro}[Scale=MatchLowercase,FakeStretch=0.9]

\usepackage[english]{babel}
\usepackage{scrlayer-scrpage}
\automark[section]{section}
\automark*[subsection]{}
\pagestyle{scrheadings}

%\usepackage{selnolig}
%\nolig{oe}{o|e}

\usepackage{xltabular}
\usepackage{fancyvrb}
\usepackage{blindtext}
\usepackage{hvlogos} % for Wikipedia W
\usepackage{ragged2e}

\usepackage[bibstyle=dtk]{biblatex}
\addbibresource{\jobname.bib}

\raggedbottom
\usepackage[colorlinks,linktocpage]{hyperref}
\newcommand*\cs[1]{\texttt{\textbackslash#1}}

\makeatletter% from: doc.sty
\newcommand*\GetFileInfo[1]{%
  \def\filename{#1}%
  \def\@tempb##1 ##2 ##3\relax##4\relax{%
    \def\filedate{##1}%
    \def\fileversion{##2}%
    \def\fileinfo{##3}}%
  \edef\@tempa{\csname ver@#1\endcsname}%
  \expandafter\@tempb\@tempa\relax? ? \relax\relax}
\makeatother

\GetFileInfo{xltabular.sty}

\title{Package \texttt{xltabular}%
\thanks{\url{https://github.com/rolfn/xltabular}}\\--\\ \normalsize \fileversion\ (\filedate)}
\author{Rolf Niepraschk\thanks{Rolf.Niepraschk@gmx.de} \and Herbert Voß\thanks{hvoss@tug.org}}
\begin{document}
\maketitle
\tableofcontents

\section{Introduction}

The package \texttt{ltablex} supports an environment \texttt{tabularx} with 
the possiblity of pagebreaks and the same header/footer definitions as in
the package \texttt{longtable}. But the old bevaviour of the \texttt{tabularx}
environment is no more possible.

The package \texttt{xltabular} defines an environment of the same name
which behaves like a \texttt{tabularx} as a \texttt{longtable}. In short:
it is a \texttt{longtable} with the column specifier \texttt{X}. But it leaves the 
\texttt{tabularx} environment with its original behaviour. 

The syntax of the environment is similiar to the one of \texttt{tabularx}:

\begin{verbatim}
\begin{xltabular}[hPos]{width}{ l X ...}
\caption{...\label{...}}\\
tabular body
\end{xltabular}
\end{verbatim}

For the optional argument \texttt{hPos} (horizontal position) the values \texttt{l}, 
\texttt{r} and \texttt{c} (default)
are possible. The environment \texttt{xltabular} uses \texttt{longtable} for
typesetting the tabular. The original \texttt{longtable} increases the counter
\texttt{table} in any case, with or without a caption. This is not the case
with \texttt{xltabular}: no \cs{caption}, no counter change.




\section{Examples}

\newcommand\B{Some nonsense text which fills at least one line
	of a tabular called nonsense tabular with text.\\}
\newcommand\BB{\B\B\B\B\B\B\B\B\B\B\B\B\B\B\B}
\newcommand\BBB{\BB\BB\BB\BB}

\listoftables

\markboth{Examples}{}
\vspace{1cm}


\subsection{The standard \texttt{tabularx} environment}

\begin{Verbatim}[frame=lines]
\captionaboveof{table}{Using the original \texttt{tabularx} environment: 
   a pagebreak not possible! This is the reason why the tabular moves 
   to the next page.}
\begin{tabularx}{0.75\linewidth}{@{} lX @{}}
left column & \blindtext\\
l           & \blindtext\\
left column & \blindtext\\
l           & \blindtext\\
\end{tabularx}
\end{Verbatim}

\bigskip
\captionaboveof{table}{Using the original \texttt{tabularx} environment: 
	a pagebreak not possible! This is the reason why the tabular moves 
	to the next page.}
\begin{tabularx}{0.75\linewidth}{@{} lX @{}}
	left column & \blindtext\\
	l           & \blindtext\\
	left column & \blindtext\\
	l           & \blindtext\\
\end{tabularx}


%\begin{tabularx}{\linewidth}{@{}X@{}}
%\captionaboveof{table}{Another example for the original \texttt{tabularx} environment: 
%%%pagebreak not possible!}\\
%\BBB
%\end{tabularx}

\clearpage

\subsection{The new environment \texttt{xltabular}}
For the demonstration we use a tabular with only one X column. There is no difference
to an example with more or other column specifiers. However, there should be at least
one \texttt{X} column specifier; it makes no real sense otherwise.

\begin{Verbatim}[frame=lines]
\begin{xltabular}{\linewidth}{@{} X @{}}
\caption{the \texttt{xltabular} environment with \texttt{longtable} 
         property}\\[\belowcaptionskip]
\BB\\
Now a manuel \cs{pagebreak} is used and on the
following page an automatic pagebreak is done by \texttt{longtable}.\\

\pagebreak

\BBB
\end{xltabular}
\end{Verbatim}

\begin{xltabular}{\linewidth}{@{} X @{}}
\caption{the \texttt{xltabular} environment with \texttt{longtable} 
	property}\\[\belowcaptionskip]
\BB\\
Now a manuel \cs{pagebreak} is used and on the
following page an automatik pagebreak is done by \texttt{longtable}.\\

\pagebreak

\BBB
\end{xltabular}

The command \cs{BBB}  does nothing else then printing a lot of lines
with nonsense.


\setcapwidth{0.75\linewidth}

\begin{Verbatim}[frame=lines]
\begin{xltabular}[l]{0.75\linewidth}{@{} lX @{}}
\caption{The \texttt{xltabular} environment with \texttt{longtable} property 
         and left aligned by setting the optional argument.}\\[\belowcaptionskip]\hline
Left column & At first a line with some nonsense text to show how long this line is. 
              The caption has the same width as the text.\\
left column & \blindtext\\
left column & \blindtext\\
left column & \blindtext\\
left column & \blindtext\\\hline
\end{xltabular}
\end{Verbatim}



\begin{xltabular}[l]{0.75\linewidth}{@{} lX @{}}
	\caption{The \texttt{xltabular} environment with \texttt{longtable} property 
		and left aligned by setting the optional argument.}\\[\belowcaptionskip]\hline
	Left column & At first a line with some nonsense text to show how long this line is. 
	The caption has the same width as the text.\\
	left column & \blindtext\\
	left column & \blindtext\\
	left column & \blindtext\\
	left column & \blindtext\\\hline
\end{xltabular}




\subsection{\texttt{tabularx} without a caption}

Here comes another \texttt{tabularx} but without a caption. It should
\emph{not} increase the \texttt{table} counter. The following tabular
in the last section should have the correct number!

\begin{tabularx}{\linewidth}{@{} >{\ttfamily}l>{\RaggedRight}X @{}}\hline
.ext & Beschreibung\\\hline
 aux & (auxiliary) Hilfsdatei, enthält Querverweise usw. \\ 
 bbl & (bibliography) Hilfsdatei, enthält die Einträge für die Literatur\\ 
 bib & (bibtex) enthält die Literaturdaten\\ 
 blg & (bibliography log) enthält die Ausgaben eines Bib\TeX-Laufs\\ 
 cfg & (config) Konfigurationsdatei\\ 
 clo & (class options) Definitionen für die Dokumentenklasse und die entsprechenden
         Klassen-Optionen\\\hline 
\end{tabularx}


\subsection{\texttt{xltabular} with the star version of caption}

Here comes another \texttt{xltabular} with \cs{caption*}. It should
\emph{not} increase the \texttt{table} counter. The following tabular
in the next section should have the correct number!

\begin{xltabular}{\linewidth}{@{} >{\ttfamily}l>{\RaggedRight}X @{}}
\caption*{A caption as star version: no number no entry in the List of Tables}\\\hline
.ext & Beschreibung\\\hline
 aux & (auxiliary) Hilfsdatei, enthält Querverweise usw. \\ 
 bbl & (bibliography) Hilfsdatei, enthält die Einträge für die Literatur\\ 
 bib & (bibtex) enthält die Literaturdaten\\ 
 blg & (bibliography log) enthält die Ausgaben eines Bib\TeX-Laufs\\ 
 cfg & (config) Konfigurationsdatei\\ 
 clo & (class options) Definitionen für die Dokumentenklasse und die entsprechenden
         Klassen-Optionen\\\hline 
\end{xltabular}



\subsection{\texttt{xltabular} with header/footer definitions}

\begin{Verbatim}[frame=lines]
\begin{xltabular}{\linewidth}{@{} >{\ttfamily}l >{\RaggedRight}X @{}}
\caption{File endings (only for a demonstration \ldots)}\\\hline
.ext & Beschreibung\\\hline
\endfirsthead
\multicolumn{2}{@{}l}{\ldots\ \small Fortsetzung}\\\hline
.ext & Beschreibung\\\hline
\endhead
\hline
\multicolumn{2}{r@{}}{\small Fortsetzung\ldots}\\
\endfoot
\hline
\endlastfoot
aux &  (auxiliary) Hilfsdatei, enthält Querverweise usw. \\ 
bbl &  (bibliography) Hilfsdatei, enthält die Einträge für die Literatur\\ 
bib &  (bibtex) enthält die Literaturdaten\\ 
blg &  (bibliography log) enthält die Ausgaben eines Bib\TeX-Laufs\\ 
cfg &  (config) Konfigurationsdatei\\ 
clo &  (class options) Definitionen für die Dokumentenklasse und die entsprechenden
[...]
Klassen-Optionen\\ 
\end{xltabular}
\end{Verbatim}

\begin{xltabular}{\linewidth}{@{} >{\ttfamily}l >{\RaggedRight}X @{}}
\caption{File endings (only for a demonstration \ldots)}\\\hline
.ext & Beschreibung\\\hline
\endfirsthead
\multicolumn{2}{@{}l}{\ldots\ \small Fortsetzung}\\\hline
.ext & Beschreibung\\\hline
\endhead
\hline
\multicolumn{2}{r@{}}{\small Fortsetzung\ldots}\\
\endfoot
\hline
\endlastfoot
 aux &  (auxiliary) Hilfsdatei, enthält Querverweise usw. \\ 
 bbl &  (bibliography) Hilfsdatei, enthält die Einträge für die Literatur\\ 
 bib &  (bibtex) enthält die Literaturdaten\\ 
 blg &  (bibliography log) enthält die Ausgaben eines Bib\TeX-Laufs\\ 
 cfg &  (config) Konfigurationsdatei\\ 
 clo &  (class options) Definitionen für die Dokumentenklasse und die entsprechenden
  Klassen-Optionen\\ 
 aux &  (auxiliary) Hilfsdatei, enthält Querverweise usw. \\ 
 bbl &  (bibliography) Hilfsdatei, enthält die Einträge für die Literatur\\ 
 bib &  (bibtex) enthält die Literaturdaten\\ 
 blg &  (bibliography log) enthält die Ausgaben eines Bib\TeX-Laufs\\ 
 cfg &  (config) Konfigurationsdatei\\ 
 clo &  (class options) Definitionen für die Dokumentenklasse und die entsprechenden
  Klassen-Optionen\\ 
 aux &  (auxiliary) Hilfsdatei, enthält Querverweise usw. \\ 
 bbl &  (bibliography) Hilfsdatei, enthält die Einträge für die Literatur\\ 
 bib &  (bibtex) enthält die Literaturdaten\\ 
 blg &  (bibliography log) enthält die Ausgaben eines Bib\TeX-Laufs\\ 
 cfg &  (config) Konfigurationsdatei\\ 
 clo &  (class options) Definitionen für die Dokumentenklasse und die entsprechenden
  Klassen-Optionen\\ 
 aux &  (auxiliary) Hilfsdatei, enthält Querverweise usw. \\ 
 bbl &  (bibliography) Hilfsdatei, enthält die Einträge für die Literatur\\ 
 bib &  (bibtex) enthält die Literaturdaten\\ 
 blg &  (bibliography log) enthält die Ausgaben eines Bib\TeX-Laufs\\ 
 cfg &  (config) Konfigurationsdatei\\ 
 clo &  (class options) Definitionen für die Dokumentenklasse und die entsprechenden
  Klassen-Optionen\\ 
\end{xltabular}

\subsection{Use of \texttt{xltabular} in an own environment}

\begin{Verbatim}[frame=lines]
\documentclass{article}
\usepackage[table]{xcolor}
\usepackage{xltabular}

\usepackage{blindtext}

\newenvironment{Merksatz}[1][black]%
  {\xltabular{\linewidth}%
      { !{\color{#1!30}\vrule width 5mm}
        X
        @{}
       }
    \cellcolor{#1!15}%
  }
  {\endxltabular}


\begin{document}

\blindtext\blindtext

\begin{Merksatz}
Hier kommt ein Merksatz \\
\blindtext \\ \blindtext\\ \blindtext
\end{Merksatz}

\blindtext

\end{document}
\end{Verbatim}

\nocite{*}
\printbibliography

\end{document}

